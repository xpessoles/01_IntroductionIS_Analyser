\documentclass[10pt]{article}
\input{style/coursHeadings}
\input{style/programHeadings}
\input{style/macros_SII}
\input{style/macros_Titres}
\input{style/macros_Frames}

%Si le boolen xp est vrai : compilation pour xabi
%Sinon compilation Damien
\newboolean{xp}
\setboolean{xp}{true}

\newboolean{prof}
\setboolean{prof}{true}



\usepackage[%
    pdftitle={Systèmes hydrauliques et pneumatiques},
    pdfauthor={Xavier Pessoles},
    colorlinks=true,
    linkcolor=blue,
    citecolor=magenta]{hyperref}


\def\discipline{Sciences Industrielles de l'Ingénieur}
\def\xxtitre{\ifthenelse{\boolean{xp}}{
CI 1 : Analyse des systèmes pluritechniques et multiphysiques -- Initiation à l'Ingénierie Système}{}}

\def\xxsoustitre{\ifthenelse{\boolean{xp}}{
Chapitre 6 -- Analyse des systèmes hydrauliques et pneumatiques}{
Partie  -- }}

\def\xxauteur{\ifthenelse{\boolean{xp}}{
Xavier \textsc{Pessoles}}{}}

\def\xxpied{\ifthenelse{\boolean{xp}}{
CI 1 : Analyse des systèmes pluritechniques et multiphysiques\\
Ch. 6 : Systèmes hydrauliques et pneumatiques -- Applications}{
\xxtitre}}

\def\xxcathegorie{\ifthenelse{\boolean{xp}}{
2013 -- 2014 \\
Xavier \textsc{Pessoles}}{
Informatique - Cours}}

%---------------------------------------------------------------------------


\begin{document}

\ifthenelse{\boolean{xp}}{\input{style/enteteXP}}{\input{style/enteteDI}}

\begin{center}
\textit{\Large{Exercices d'application}}
\end{center}

\subsection*{Exercice 1 : Schéma pneumatique de la capsuleuse de bocaux}

On donne la chaine fonctionnelle permettant de monter et descendre l'ensemble ventouse de la capsuleuse, le schéma pneumatique ainsi que le schéma électrique de câblage de l'automate.

\begin{center}
\includegraphics[width=.95\textwidth]{images/CECF}
\end{center}

\begin{minipage}[b]{.48\linewidth}
\begin{center}
\includegraphics[width=\textwidth]{images/AlimPneu}

\textit{Alimentation pneumatique}
\end{center}
\end{minipage}\hfill
\begin{minipage}[b]{.48\linewidth}
\begin{center}
\includegraphics[width=\textwidth]{images/Pneu02}

\textit{Système de blocage du bocal et d'aspiration de la capsule}
\end{center}
\end{minipage}

\vspace{.5cm}

\begin{center}
\includegraphics[width=.95\textwidth]{images/Pneu01}

\textit{Alimentation pneumatique}
\end{center}


\begin{center}
\includegraphics[width=.95\textwidth]{images/TSX3710_in}

\textit{Schéma électrique correspondant aux entrées de l'automate}
\end{center}

\begin{center}
\includegraphics[width=.95\textwidth]{images/TSX3710_out}

\textit{Schéma électrique correspondant aux sorties de l'automate}
\end{center}

\newpage{}

\subsubsection*{Bloc d'alimentation}
\subparagraph{}
\textit{Donner la désignation des constituants du bloc d'alimentation pneumatique.}

\subsubsection*{Système de montée et descente de la capsule}
\subparagraph{}
\textit{Identifier l'actionneur et les pré actionneurs associés à la montée et la descente de la tête. Donner leur désignation.}

\subparagraph{}
\textit{Expliquer le lien existant entre l'automate et le préactionneur. }

\subparagraph{}
\textit{Sur le schéma électrique correspondant aux entrées de l'automate, expliquer à quoi correspondent les entrées \%I1.4 et \%I1.5.}


\subparagraph{}
\textit{Que se passe-t-il quand la sortie \%Q2.5 de l'automate est activée ?}



\subsection*{Exercice 2 : Schéma hydraulique du pilote automatique de voilier}
\setcounter{subparagraph}{0}
\vspace{.5cm}
\begin{center}
\includegraphics[width=.4\textwidth]{images/GroupeHydraulique}
\hfill
\includegraphics[width=.4\textwidth]{images/GroupeHydraulique}
\end{center}

\subparagraph{}
\textit{Donner la désignation de chacun des composants hydrauliques.}

\subparagraph{}
\textit{Analyser ce schéma dans les cas d’utilisation  suivants :
\begin{itemize}
\item le vérin sort à droite;
\item le vérin sort à gauche;
\item l’effort sur le vérin devient trop important;
\item le vérin est immobile mais doit rester en place sous les efforts;
\item le vérin doit se déplacer librement car le safran est sous commande manuelle (le volant du bateau).
\end{itemize}}

\subparagraph{}
\textit{Proposer un schéma cinématique de la pompe (on précise qu'il s'agit d'une pompe à pistions radiaux).}
\end{document}


